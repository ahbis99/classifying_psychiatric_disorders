%%%%%%%%%%%%%%%%%%%%%%%%%%%%%%%%%%%%%%%%%
% FRI Data Science_report LaTeX Template
% Version 1.0 (28/1/2020)
% 
% Jure Demšar (jure.demsar@fri.uni-lj.si)
%
% Based on MicromouseSymp article template by:
% Mathias Legrand (legrand.mathias@gmail.com) 
% With extensive modifications by:
% Antonio Valente (antonio.luis.valente@gmail.com)
%
% License:
% CC BY-NC-SA 3.0 (http://creativecommons.org/licenses/by-nc-sa/3.0/)
%
%%%%%%%%%%%%%%%%%%%%%%%%%%%%%%%%%%%%%%%%%


%----------------------------------------------------------------------------------------
%	PACKAGES AND OTHER DOCUMENT CONFIGURATIONS
%----------------------------------------------------------------------------------------
\documentclass[fleqn,moreauthors,10pt]{ds_report}
\usepackage[english]{babel}
\graphicspath{{fig/}}




%----------------------------------------------------------------------------------------
%	ARTICLE INFORMATION
%----------------------------------------------------------------------------------------

% Header
\JournalInfo{FRI Data Science Project Competition 2020}

% Interim or final report
\Archive{Interim report} 
%\Archive{Final report} 

% Article title
\PaperTitle{Classifying Psychiatric Disorders} 

% Authors (student competitors) and their info
\Authors{Ahmet Çalış, Manfred Gonzalez-Hernandez, and Joaquín Figueira}

% Advisors
\affiliation{\textit{Advisors: Prof. Dr. Jure Demšar}}

% Keywords
\Keywords{Data Science, Psychiatry, fMRI, cGCN, B-SNIP, GBC, FC}
\newcommand{\keywordname}{Keywords}


%----------------------------------------------------------------------------------------
%	ABSTRACT
%----------------------------------------------------------------------------------------
\Abstract{

In recent times the need for a bioneurological basis to treat psychosis spectrum disorders has become apparent, as the clinical treatment of these disorders still relies on self-reported behavioral analysis of the patients which has proven to be inaccurate. In this work we use dimensionality reduced data from fMRI scans to predict the disorders of a set of 711 psychosis spectrum disorder patients and control group, with the ultimate goal of finding a small latent space of the fMRI data in which the different disorders and their symptoms are clearly distinguishable. The preliminary results suggest that it's possible to accurately differentiate a patient from the control group, but predicting specific disorders and symptomatic scales has proven unsuccessful. 
}

%----------------------------------------------------------------------------------------

\begin{document}

% Makes all text pages the same height
\flushbottom 

% Print the title and abstract box
\maketitle 

% Removes page numbering from the first page
\thispagestyle{empty} 

%----------------------------------------------------------------------------------------
%	ARTICLE CONTENTS
%----------------------------------------------------------------------------------------

\section*{Introduction}

    Given the high levels of complexity of the human brain and its associated afflictions, diagnosing and treating psychiatric disorders is a notoriously challenging endeavor. This is specially true in the psychosis spectrum of disorders (PSD), where the literature shows that, due to its greater symptomatic variability, there is a pressing need for more complex and individualized patient care.

    One of the biggest impediments to this sort of treatment is the fact that traditional clinical approaches used to diagnose and treat these disorders relay on behavioral and self reported observations of the patients symptoms, such as the Positive and Negative Syndrome Scale (PANSS). Although a successful neural mapping across canonical schizofrenia (SZP) symptoms has been found in previous works \cite{Chen2020}, the field still lacks an accurate neural mapping for the full spectrum of pyschosis disorders. Finding such a mapping could be used to provide more accurate forms of treatments and diagnosis with a sound biomolecular basis.

    In this context, this project aims to bridge this behavioral-neurological gap by developing Machine Learning (ML) approaches to relate different patient fMRI scans with their corresponding PANSS scores, using data obtained by the Bipolar and Schizophrenia Network for Intermediate Phenotypes consortium (B-SNIP)\cite{Clementz2016}. The data consists of a set of PANSS scores and resting state fMRIs scans from patients diagnosed with schizophrenia (SZP), schizoaffective disorder (SADP) and bipolar disorder with psychosis (BPP), and a control group (CON). The ultimate objective of this work is to build a classifier that can accurately predict the specific disorder of a patient using their corresponding fMRI data, and use it to find a small latent space representation of the neural data that highly correlates to PSD symptoms.
    

\iffalse    
    Using this data we can also build Functional Connectivity representations between brain regions, which has been shown to find important patterns in brain activity \cite{Matkovič_2024, Wang2021-ts}. One of such representations are connectivity-based graphs convolutional networks (cGCN) architecture for fMRI analysis, allowing the extraction of spatial features from connectomic neighborhoods, showing effectiveness in individual identification and classification of ASD patients in \cite{Wang2021-ts}. The proposed architecture was applied to supervised classification experiments using rs-fMRI data from the Human Connectome Project, showcasing its performance in identifying subjects based on their rs-fMRI data.

        In addition to the patient and control groups PANSS scores, the data is composed of a set of time series of 3D-images for each patient of different regions of the brain, from which we obtain a set of correlation-based connectivity features by finding the correlations between regions across time. We used two of such connectivity feature extraction methods, which has been proposed as imaging markers for several psychiatric disorders \cite{Kraus2020-ut}. Using this data we implemented several Machine learning models to build a binary classifier of sick or healthy patients. Then we escalated the analysis to a multi-label classifier of the different diagnosis of the patients.
\fi


%------------------------------------------------

\section*{Methods}
\label{sec:methods}

\subsection*{Data preprocessing and dimensionality reduction}

An fMRI scan records the brain activity of a patient through blood oxygenation measurements. In these measurements the fMRI scanner intrinsically divides the brain into different regions (formally called voxels) according to a pre-defined spatial resolution, which tipycally results in a subdivision of the brain in around 90,000 voxels. The result is a time series of 3-D images of blood oxygenation levels for each voxel sampled with an approximate time resolution of 1.5 seconds during 20-30 minutes, which all combined amounts to very large data volumens for even a single patient.  

As the high dimensionality of this data representation would probably lead to high degrees of overfitting and excessively high runtimes, we used several dimensionality reduction techniques to make it more digestible. First, we merged voxels into regions of interest (ROIs), for which we used two parcellations: the Glasser parcellation which has 718 ROIs and the Cole-Anticevic parcellations whith 12 ROIs.

With this first reduction we obtained a time series of images with more compact resolutions. However, the data volume was still intractable due to the size of the time dimension of the series. To tackle this we used two other techniques for time dimensionality reduction: Functional Connectome (FC) and Global Brain Connectivity (GBC). The first approach, FC, measures the total correlation between each pair of ROIs across time. In other words, for a parcellation with $r$ ROIs, the data is reduced to a $\mathbb{R}^{r\times r}$ matrix. The second approach, GBC, computes the average correlation between one ROI and all the others across time, resulting in a $r$ dimensional vector. 

\subsection*{Models}
%These are all the traditional machine learning algorithms we used over the tabular data
The initial solution we implemented for the first part of the competition was using the GBC data with both mentioned parcellations to predict the patients' health status (i.e whether the person has a disorder or not) and specific disorder. We trained 6 different classifiers on these data for each parcellation: three classifiers for binary health status classification, three classifiers for multi-label disorder classification. 

To build these classifiers we used three learners: random forests, XGBoost and MLPs. As random forests are known to perform relatively well with little parameter and feature extraction procedures, we decided to use them as a general baseline. Since we're working in the medical domain, explainability is key, which is the reason why we decided to use XGBoost as our main classification model, as it provides both decent performance and explainability. Additionally, it can perform well in low sample environments. To test more complex feature interaction models, we decided to use Multi Layer Perceptrons (MLP). 

Finally, we performed hyper-paramer tunning using a train-test split of the data (with test size of 20\%) and computed the models accuracy on each of the splits. 

\section*{Results} \label{results}

In this section we present the preliminary results of the models explained in the previous section. The results for the 718 ROI Glasser parcellation are presented in tables \ref{tab:glassier_binary_classification} and \ref{tab:glasser_multiclass_classification}, while the results for the 12 ROI Cole-Anticevic parcellation are presented in tables \ref{tab:cole_binary_classification} and \ref{tab:cole_multiclass_classification}.

% Random Forests
% XGBoost
% MLP


\begin{table}[h!]
\centering
\begin{tabular}{|l|c|c|c|}
\hline
\textbf{Metric} & \textbf{XGBoost} & \textbf{Random Forest}& \textbf{MLP} \\ \hline
Accuracy (Tr) & 1.0 & 1.0 & 1.0 \\ \hline
Accuracy (Ts) & 0.984 & 0.977 & 0.98 \\ \hline
\end{tabular}
\caption{\textbf{Binary classification results using the Glasser parcellation data}. The top row shows prediction accuracy on the training set (Tr), while the bottom row shows prediciton accuracy o the test set (Ts).}
\label{tab:glassier_binary_classification}
\end{table}

\begin{table}[h!]
\centering
\begin{tabular}{|l|c|c|c|}
\hline
\textbf{Metric} & \textbf{XGBoost} & \textbf{Random Forest} & \textbf{MLP} \\ \hline
Accuracy (Tr) & 1.0 & 1.0 & 1.0 \\ \hline
Accuracy (Ts) & 0.578 & 0.586 & 0.55 \\ \hline
\end{tabular}
\caption{\textbf{Multiclass classifications results using the Glasser parcellation data}. The top row shows prediction accuracy on the training set (Tr), while the bottom row shows prediciton accuracy o the test set (Ts).}
\label{tab:glasser_multiclass_classification}
\end{table}

\begin{table}[h!]
\centering
\begin{tabular}{|l|c|c|c|}
\hline
\textbf{Metric} & \textbf{XGBoost} & \textbf{Random Forest}& \textbf{MLP} \\ \hline
Accuracy (Tr) & 1.0 & 1.0 & 0.82 \\ \hline
Accuracy (Ts) & 0.648 & 0.617 & 0.62 \\ \hline
\end{tabular}
\caption{\textbf{Binary classification results using the Cole-Antisevic parcellation data}. The top row shows prediction accuracy on the training set (Tr), while the bottom row shows prediciton accuracy o the test set (Ts).}
\label{tab:cole_binary_classification}
\end{table}

\begin{table}[h!]
\centering
\begin{tabular}{|l|c|c|c|}
\hline
\textbf{Metric} & \textbf{XGBoost} & \textbf{Random Forest} & \textbf{MLP} \\ \hline
Accuracy (Tr) & 1.0 & 1.0 & 0.39 \\ \hline
Accuracy (Ts) & 0.367 & 0.312 & 0.35 \\ \hline
\end{tabular}
\caption{\textbf{Multiclass classifications results using the Cole-Antisevic parcellation data}. The top row shows prediction accuracy on the training set (Tr), while the bottom row shows prediciton accuracy o the test set (Ts).}
\label{tab:cole_multiclass_classification}
\end{table}

\section*{Discussion}
The results so far suggest that the problem of identifying the binary health status of a patient using fMRI data is solvable, as already suggested in the literature. However, the classification of specific disorders seems to be a more complex problem, and it will require more involved data processing and modeling to achieve good results. The results also shows high amounts of overfitting for this last task.  Finally, the 12 region data parcellation seems to compress the structural information of the brain activation data too excessively, obscuring the relationship between the disorders and their neurological expression resulting in very poor classification accuracies.

\section*{Future Work}
In the upcoming stage of our research, we will pursue the development of a graph-based framework to enrich feature extraction methods and augment multi-class classification accuracy. By employing a k-Nearest Neighbors graph that treats brain regions as nodes and functional connectivity as edges, we plan to refine the graph structure using the hyperparameter k to judiciously prune and maintain essential connections. Under this architecture, these graphs could be the initial stage of graph Convolutional Neural Networks (gCNN) for classification \cite{Wang2021-ts}.

Complementing this, we aspire to design an autoencoder that leverages Global Brain Connectivity (GBC) data to distill a 3D latent space, which will be scrutinized for potential correlations with classification targets. The dissection of this latent space is anticipated to uncover underlying data structures that correlate with diagnostic categories, offering a novel avenue to boost classification performance.
%------------------------------------------------



%----------------------------------------------------------------------------------------
%	REFERENCE LIST
%----------------------------------------------------------------------------------------
\bibliographystyle{unsrt}
\bibliography{report}


\end{document}